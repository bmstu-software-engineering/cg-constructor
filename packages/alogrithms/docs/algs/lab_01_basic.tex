\documentclass[a4paper,12pt]{article}

\usepackage[utf8]{inputenc}
\usepackage[T2A]{fontenc}
\usepackage[russian]{babel}
\usepackage{amsmath}
\usepackage{amssymb}
\usepackage{graphicx}
\usepackage{algorithm}
\usepackage{algpseudocode}
\usepackage{geometry}
\usepackage{hyperref}
\usepackage{tikz}

\geometry{left=2cm, right=2cm, top=2cm, bottom=2cm}

\title{Документация базового алгоритма визуализации точек}
\author{Базовый вариант}
\date{\today}

\begin{document}

\maketitle

\tableofcontents

\newpage

\section{Введение}

Данный документ содержит описание базового алгоритма визуализации точек. Алгоритм реализован в классе \texttt{AlgorithmL01VBasic} и является базовым примером для демонстрации работы с точками на плоскости.

\section{Постановка задачи}

\subsection{Входные данные}
На вход алгоритму подается множество точек на плоскости $P = \{p_1, p_2, \ldots, p_n\}$, где $p_i = (x_i, y_i)$.

\subsection{Выходные данные}
Результатом работы алгоритма является визуализация входного множества точек без дополнительных преобразований, а также текстовая информация о количестве точек и их координатах.

\subsection{Формальная постановка}
Пусть $P = \{p_1, p_2, \ldots, p_n\}$ — множество точек на плоскости. Требуется отобразить это множество точек на графике и предоставить текстовую информацию о количестве точек и их координатах.

\section{Описание алгоритма}

\subsection{Общая схема алгоритма}

Алгоритм состоит из следующих основных шагов:
\begin{enumerate}
    \item Получение множества точек из модели данных.
    \item Формирование текстовой информации о количестве точек и их координатах.
    \item Возвращение результата для визуализации.
\end{enumerate}

\subsection{Псевдокод}

\begin{algorithm}
\caption{Базовый алгоритм визуализации точек}
\begin{algorithmic}[1]
\Procedure{VisualizePoints}{$points$}
    \State $markdownInfo \gets$ FormatMarkdownInfo($points$)
    \State \textbf{return} ViewerResultModel($points$, [], $markdownInfo$)
\EndProcedure
\end{algorithmic}
\end{algorithm}

\begin{algorithm}
\caption{Форматирование текстовой информации}
\begin{algorithmic}[1]
\Procedure{FormatMarkdownInfo}{$points$}
    \State $info \gets$ "## Результаты базового алгоритма\n\n"
    \State $info \gets info +$ "### Количество точек\n"
    \State $info \gets info +$ "Всего точек: **$|points|$**\n\n"
    \State $info \gets info +$ "### Координаты точек\n"
    \State $info \gets info +$ FormatPointsList($points$)
    \State \textbf{return} $info$
\EndProcedure
\end{algorithmic}
\end{algorithm}

\begin{algorithm}
\caption{Форматирование списка точек}
\begin{algorithmic}[1]
\Procedure{FormatPointsList}{$points$}
    \State $result \gets$ пустая строка
    \For{$i \gets 0$ to $|points| - 1$}
        \State $point \gets points[i]$
        \State $result \gets result +$ "- Точка $(i + 1)$: $(point.x, point.y)$\n"
    \EndFor
    \State \textbf{return} $result$
\EndProcedure
\end{algorithmic}
\end{algorithm}

\section{Анализ алгоритма}

\subsection{Временная сложность}

Временная сложность алгоритма определяется следующими факторами:
\begin{itemize}
    \item Получение множества точек из модели данных: $O(1)$
    \item Формирование текстовой информации: $O(n)$, где $n$ — количество точек
    \item Возвращение результата для визуализации: $O(1)$
\end{itemize}

Таким образом, общая временная сложность алгоритма составляет $O(n)$.

\subsection{Пространственная сложность}

Пространственная сложность алгоритма определяется следующими факторами:
\begin{itemize}
    \item Хранение исходного множества точек: $O(n)$
    \item Хранение текстовой информации: $O(n)$
\end{itemize}

Таким образом, общая пространственная сложность алгоритма составляет $O(n)$.

\section{Примеры}

\subsection{Пример 1}

\textbf{Входные данные:}
\begin{itemize}
    \item Множество точек: $\{(0, 0), (1, 1), (2, 2), (3, 3)\}$
\end{itemize}

\textbf{Результат:}
\begin{itemize}
    \item Визуализация точек на графике
    \item Текстовая информация:
    \begin{verbatim}
## Результаты базового алгоритма

### Количество точек
Всего точек: **4**

### Координаты точек
- Точка 1: (0, 0)
- Точка 2: (1, 1)
- Точка 3: (2, 2)
- Точка 4: (3, 3)
    \end{verbatim}
\end{itemize}

\begin{figure}[h]
\centering
\begin{tikzpicture}[scale=1]
    % Оси координат
    \draw[->] (-0.5,0) -- (3.5,0) node[right] {$x$};
    \draw[->] (0,-0.5) -- (0,3.5) node[above] {$y$};
    
    % Точки
    \fill (0,0) circle (0.1) node[below left] {(0,0)};
    \fill (1,1) circle (0.1) node[above right] {(1,1)};
    \fill (2,2) circle (0.1) node[above right] {(2,2)};
    \fill (3,3) circle (0.1) node[above right] {(3,3)};
\end{tikzpicture}
\caption{Пример результата работы алгоритма}
\end{figure}

\end{document}
