\documentclass[a4paper,12pt]{article}

\usepackage[utf8]{inputenc}
\usepackage[T2A]{fontenc}
\usepackage[russian]{babel}
\usepackage{amsmath}
\usepackage{amssymb}
\usepackage{graphicx}
\usepackage{algorithm}
\usepackage{algpseudocode}
\usepackage{geometry}
\usepackage{hyperref}
\usepackage{tikz}

\geometry{left=2cm, right=2cm, top=2cm, bottom=2cm}

\title{Документация алгоритма поиска линии с максимальным углом между треугольниками с тупыми углами}
\author{Вариант 40}
\date{\today}

\begin{document}

\maketitle

\tableofcontents

\newpage

\section{Введение}

Данный документ содержит описание алгоритма поиска линии с максимальным углом между вершинами с тупыми углами в двух множествах треугольников. Алгоритм реализован в классе \texttt{AlgorithmL01V40} и является частью лабораторной работы 01, вариант 40.

\section{Постановка задачи}

\subsection{Входные данные}
На вход алгоритму подаются два множества точек на плоскости:
\begin{itemize}
    \item Первое множество: $P_1 = \{p_{1,1}, p_{1,2}, \ldots, p_{1,n}\}$, где $p_{1,i} = (x_{1,i}, y_{1,i})$ и $n \geq 3$.
    \item Второе множество: $P_2 = \{p_{2,1}, p_{2,2}, \ldots, p_{2,m}\}$, где $p_{2,j} = (x_{2,j}, y_{2,j})$ и $m \geq 3$.
\end{itemize}

\subsection{Выходные данные}
Результатом работы алгоритма является:
\begin{itemize}
    \item Треугольник $T_1 = \{A_1, B_1, C_1\}$ из первого множества, имеющий тупой угол при одной из вершин (обозначим её $V_1$).
    \item Треугольник $T_2 = \{A_2, B_2, C_2\}$ из второго множества, имеющий тупой угол при одной из вершин (обозначим её $V_2$).
    \item Линия $L = (V_1, V_2)$, соединяющая вершины с тупыми углами.
\end{itemize}

Линия $L$ должна образовывать максимальный угол с осью абсцисс среди всех возможных пар треугольников с тупыми углами из первого и второго множеств.

\subsection{Формальная постановка}
Пусть $T_1 = \{A_1, B_1, C_1\}$ — треугольник с вершинами из множества $P_1$, имеющий тупой угол при вершине $V_1 \in \{A_1, B_1, C_1\}$.

Пусть $T_2 = \{A_2, B_2, C_2\}$ — треугольник с вершинами из множества $P_2$, имеющий тупой угол при вершине $V_2 \in \{A_2, B_2, C_2\}$.

Пусть $\alpha(V_1, V_2)$ — угол между линией $(V_1, V_2)$ и осью абсцисс.

Требуется найти такие треугольники $T_1$ и $T_2$ с тупыми углами при вершинах $V_1$ и $V_2$ соответственно, для которых величина $\alpha(V_1, V_2)$ максимальна.

\section{Описание алгоритма}

\subsection{Общая схема алгоритма}

Алгоритм состоит из следующих основных шагов:
\begin{enumerate}
    \item Проверка на достаточное количество точек в обоих множествах (не менее 3).
    \item Генерация всех возможных треугольников из первого множества точек.
    \item Генерация всех возможных треугольников из второго множества точек.
    \item Нахождение треугольников с тупыми углами в первом множестве.
    \item Нахождение треугольников с тупыми углами во втором множестве.
    \item Для каждой пары треугольников с тупыми углами:
    \begin{enumerate}
        \item Вычисление угла между линией, соединяющей вершины с тупыми углами, и осью абсцисс.
        \item Обновление максимального угла и соответствующих треугольников.
    \end{enumerate}
    \item Формирование результата для визуализации.
\end{enumerate}

\subsection{Псевдокод}

\begin{algorithm}
\caption{Поиск линии с максимальным углом между треугольниками с тупыми углами}
\begin{algorithmic}[1]
\Procedure{FindMaxAngleLine}{$pointsFirst, pointsSecond$}
    \If{$|pointsFirst| < 3$ \textbf{or} $|pointsSecond| < 3$}
        \State \textbf{throw} InsufficientPointsException
    \EndIf
    
    \State $trianglesFirst \gets$ GenerateTriangles($pointsFirst$)
    \State $trianglesSecond \gets$ GenerateTriangles($pointsSecond$)
    
    \State $obtuseFirst \gets$ FindObtuseTriangles($trianglesFirst$)
    \If{$obtuseFirst$ is empty}
        \State \textbf{throw} NoObtuseAnglesException("Первое множество")
    \EndIf
    
    \State $obtuseSecond \gets$ FindObtuseTriangles($trianglesSecond$)
    \If{$obtuseSecond$ is empty}
        \State \textbf{throw} NoObtuseAnglesException("Второе множество")
    \EndIf
    
    \State $maxAngle \gets 0$
    \State $resultLine \gets null$
    \State $firstTriangle \gets null$
    \State $secondTriangle \gets null$
    \State $firstObtuseVertex \gets null$
    \State $secondObtuseVertex \gets null$
    
    \For{each $(triangle1, vertex1)$ in $obtuseFirst$}
        \For{each $(triangle2, vertex2)$ in $obtuseSecond$}
            \State $angle \gets$ CalculateAngle($vertex1, vertex2$)
            \If{$angle > maxAngle$}
                \State $maxAngle \gets angle$
                \State $resultLine \gets$ Line($vertex1, vertex2$)
                \State $firstTriangle \gets triangle1$
                \State $secondTriangle \gets triangle2$
                \State $firstObtuseVertex \gets vertex1$
                \State $secondObtuseVertex \gets vertex2$
            \EndIf
        \EndFor
    \EndFor
    
    \If{$resultLine = null$}
        \State \textbf{throw} CalculationException
    \EndIf
    
    \State \textbf{return} BuildResult($firstTriangle, secondTriangle, firstObtuseVertex, secondObtuseVertex, resultLine$)
\EndProcedure
\end{algorithmic}
\end{algorithm}

\subsection{Вспомогательные алгоритмы}

\subsubsection{Генерация треугольников}

Для генерации всех возможных треугольников из множества точек используется класс \texttt{TriangleGenerator}. Алгоритм перебирает все возможные тройки точек и создает из них треугольники.

\begin{algorithm}
\caption{Генерация треугольников}
\begin{algorithmic}[1]
\Procedure{GenerateTriangles}{$points$}
    \State $triangles \gets$ пустой список
    \For{$i \gets 0$ to $|points| - 3$}
        \For{$j \gets i + 1$ to $|points| - 2$}
            \For{$k \gets j + 1$ to $|points| - 1$}
                \State $triangle \gets$ Triangle($points[i]$, $points[j]$, $points[k]$)
                \State Добавить $triangle$ в $triangles$
            \EndFor
        \EndFor
    \EndFor
    \State \textbf{return} $triangles$
\EndProcedure
\end{algorithmic}
\end{algorithm}

\subsubsection{Поиск треугольников с тупыми углами}

Для поиска треугольников с тупыми углами используется класс \texttt{TriangleAnalyzer}. Алгоритм проверяет каждый треугольник на наличие тупого угла и возвращает список пар (треугольник, вершина с тупым углом).

\begin{algorithm}
\caption{Поиск треугольников с тупыми углами}
\begin{algorithmic}[1]
\Procedure{FindObtuseTriangles}{$triangles$}
    \State $obtuseTriangles \gets$ пустой список
    \For{each $triangle$ in $triangles$}
        \State $obtuseVertex \gets$ FindObtuseAngleVertex($triangle$)
        \If{$obtuseVertex \neq null$}
            \State Добавить $(triangle, obtuseVertex)$ в $obtuseTriangles$
        \EndIf
    \EndFor
    \State \textbf{return} $obtuseTriangles$
\EndProcedure
\end{algorithmic}
\end{algorithm}

\begin{algorithm}
\caption{Определение вершины с тупым углом}
\begin{algorithmic}[1]
\Procedure{FindObtuseAngleVertex}{$triangle$}
    \State $a \gets triangle.a$
    \State $b \gets triangle.b$
    \State $c \gets triangle.c$
    
    \State $ab \gets$ CalculateDistance($a, b$)
    \State $bc \gets$ CalculateDistance($b, c$)
    \State $ca \gets$ CalculateDistance($c, a$)
    
    \State $cosA \gets (ab^2 + ca^2 - bc^2) / (2 \cdot ab \cdot ca)$
    \If{$cosA < 0$}
        \State \textbf{return} $a$
    \EndIf
    
    \State $cosB \gets (ab^2 + bc^2 - ca^2) / (2 \cdot ab \cdot bc)$
    \If{$cosB < 0$}
        \State \textbf{return} $b$
    \EndIf
    
    \State $cosC \gets (bc^2 + ca^2 - ab^2) / (2 \cdot bc \cdot ca)$
    \If{$cosC < 0$}
        \State \textbf{return} $c$
    \EndIf
    
    \State \textbf{return} $null$
\EndProcedure
\end{algorithmic}
\end{algorithm}

\subsubsection{Вычисление угла между линией и осью абсцисс}

Для вычисления угла между линией и осью абсцисс используется класс \texttt{GeometryCalculator}. Алгоритм использует функцию $\arctan2$ для вычисления угла.

\begin{algorithm}
\caption{Вычисление угла между линией и осью абсцисс}
\begin{algorithmic}[1]
\Procedure{CalculateAngle}{$p1, p2$}
    \State $angle \gets \arctan2(p2_y - p1_y, p2_x - p1_x)$
    \If{$angle < 0$}
        \State $angle \gets angle + \pi$
    \EndIf
    \State \textbf{return} $angle$
\EndProcedure
\end{algorithmic}
\end{algorithm}

\section{Примеры}

\subsection{Пример 1}

\textbf{Входные данные:}
\begin{itemize}
    \item Первое множество точек: $\{(0, 0), (10, 0), (5, -1)\}$
    \item Второе множество точек: $\{(0, 5), (10, 5), (5, 6)\}$
\end{itemize}

\textbf{Результат:}
\begin{itemize}
    \item Первый треугольник: $\{(0, 0), (10, 0), (5, -1)\}$ с тупым углом при вершине $(5, -1)$
    \item Второй треугольник: $\{(0, 5), (10, 5), (5, 6)\}$ с тупым углом при вершине $(5, 6)$
    \item Линия: $((5, -1), (5, 6))$
    \item Угол с осью абсцисс: $90°$
\end{itemize}

\begin{figure}[h]
\centering
\begin{tikzpicture}[scale=0.5]
    % Первый треугольник (зеленый)
    \draw[green, thick] (0,0) -- (10,0) -- (5,-1) -- cycle;
    
    % Второй треугольник (синий)
    \draw[blue, thick] (0,5) -- (10,5) -- (5,6) -- cycle;
    
    % Вершины с тупыми углами (увеличенные)
    \fill[green] (5,-1) circle (0.2);
    \fill[blue] (5,6) circle (0.2);
    
    % Результирующая линия (красная)
    \draw[red, thick] (5,-1) -- (5,6);
    
    % Остальные вершины
    \fill[green] (0,0) circle (0.1);
    \fill[green] (10,0) circle (0.1);
    \fill[blue] (0,5) circle (0.1);
    \fill[blue] (10,5) circle (0.1);
\end{tikzpicture}
\caption{Пример результата работы алгоритма}
\end{figure}

\end{document}
