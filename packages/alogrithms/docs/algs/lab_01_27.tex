\documentclass[a4paper,12pt]{article}

\usepackage[utf8]{inputenc}
\usepackage[T2A]{fontenc}
\usepackage[russian]{babel}
\usepackage{amsmath}
\usepackage{amssymb}
\usepackage{graphicx}
\usepackage{algorithm}
\usepackage{algpseudocode}
\usepackage{geometry}
\usepackage{hyperref}
\usepackage{tikz}

\geometry{left=2cm, right=2cm, top=2cm, bottom=2cm}

\title{Документация алгоритма поиска треугольника с максимальной разностью точек в подтреугольниках}
\author{Вариант 27}
\date{\today}

\begin{document}

\maketitle

\tableofcontents

\newpage

\section{Введение}

Данный документ содержит описание алгоритма поиска треугольника, максимизирующего разность между количеством точек в подтреугольниках, образованных медианами. Алгоритм реализован в классе \texttt{AlgorithmL01V27} и является частью лабораторной работы 01, вариант 27.

\section{Постановка задачи}

\subsection{Входные данные}
На вход алгоритму подается множество точек на плоскости $P = \{p_1, p_2, \ldots, p_n\}$, где $p_i = (x_i, y_i)$ и $n \geq 3$.

\subsection{Выходные данные}
Результатом работы алгоритма является треугольник $T = \{A, B, C\}$, где $A, B, C \in P$, такой, что разность между максимальным и минимальным количеством точек из $P$ в подтреугольниках, образованных медианами $T$, максимальна.

\subsection{Формальная постановка}
Пусть $T = \{A, B, C\}$ — треугольник с вершинами из множества $P$. Медианы треугольника $T$ разбивают его на 6 подтреугольников: $T_1, T_2, \ldots, T_6$. Для каждого подтреугольника $T_i$ определим $count(T_i)$ — количество точек из $P$, лежащих внутри $T_i$.

Требуется найти такой треугольник $T$, для которого величина
\begin{equation}
\max_{i} count(T_i) - \min_{i} count(T_i)
\end{equation}
максимальна среди всех возможных треугольников с вершинами из $P$.

\section{Описание алгоритма}

\subsection{Общая схема алгоритма}

Алгоритм состоит из следующих основных шагов:
\begin{enumerate}
    \item Проверка на достаточное количество точек (не менее 3).
    \item Генерация всех возможных треугольников из множества точек.
    \item Для каждого треугольника:
    \begin{enumerate}
        \item Проверка, не лежат ли его вершины на одной прямой.
        \item Разбиение треугольника на 6 подтреугольников с помощью медиан.
        \item Подсчет количества точек в каждом подтреугольнике.
        \item Нахождение максимального и минимального количества точек.
        \item Вычисление разности между максимальным и минимальным количеством.
    \end{enumerate}
    \item Выбор треугольника с максимальной разностью.
    \item Формирование результата для визуализации.
\end{enumerate}

\newpage
\subsection{Псевдокод}

\begin{algorithm}

\caption{Поиск треугольника с максимальной разностью точек в подтреугольниках}
\begin{algorithmic}[1]
\Procedure{FindMaxDifferenceTriangle}{$points$}
    \If{$|points| < 3$}
        \State \textbf{throw} InsufficientPointsException
    \EndIf
    
    \State $triangles \gets$ GenerateTriangles($points$)
    \State $maxDifference \gets -1$
    \State $bestTriangle \gets null$
    \State $bestCounts \gets null$
    \State $maxIndex \gets null$
    \State $minIndex \gets null$
    
    \For{each $triangle$ in $triangles$}
        \If{IsTriangleDegenerate($triangle$)}
            \State \textbf{continue}
        \EndIf
        
        \State $subtriangles \gets$ DivideTriangleByMedians($triangle$)
        \State $counts \gets$ CountPointsInsideSubtriangles($points$, $subtriangles$)
        
        \State $maxCount \gets counts[0]$
        \State $minCount \gets counts[0]$
        \State $maxIdx \gets 0$
        \State $minIdx \gets 0$
        
        \For{$i \gets 1$ to $|counts| - 1$}
            \If{$counts[i] > maxCount$}
                \State $maxCount \gets counts[i]$
                \State $maxIdx \gets i$
            \EndIf
            \If{$counts[i] < minCount$}
                \State $minCount \gets counts[i]$
                \State $minIdx \gets i$
            \EndIf
        \EndFor
        
        \State $difference \gets maxCount - minCount$
        
        \If{$difference > maxDifference$}
            \State $maxDifference \gets difference$
            \State $bestTriangle \gets triangle$
            \State $bestCounts \gets counts$
            \State $maxIndex \gets maxIdx$
            \State $minIndex \gets minIdx$
        \EndIf
    \EndFor
    
    \If{$bestTriangle = null$}
        \State \textbf{throw} CalculationException
    \EndIf
    
    \State \textbf{return} BuildResult($bestTriangle$, $bestCounts$, $maxIndex$, $minIndex$)
\EndProcedure
\end{algorithmic}
\end{algorithm}

\newpage
\subsection{Вспомогательные алгоритмы}

\subsubsection{Генерация треугольников}

Для генерации всех возможных треугольников из множества точек используется класс \texttt{TriangleGenerator}. Алгоритм перебирает все возможные тройки точек и создает из них треугольники.

\begin{algorithm}
\caption{Генерация треугольников}
\begin{algorithmic}[1]
\Procedure{GenerateTriangles}{$points$}
    \State $triangles \gets$ пустой список
    \For{$i \gets 0$ to $|points| - 3$}
        \For{$j \gets i + 1$ to $|points| - 2$}
            \For{$k \gets j + 1$ to $|points| - 1$}
                \State $triangle \gets$ Triangle($points[i]$, $points[j]$, $points[k]$)
                \State Добавить $triangle$ в $triangles$
            \EndFor
        \EndFor
    \EndFor
    \State \textbf{return} $triangles$
\EndProcedure
\end{algorithmic}
\end{algorithm}

\subsubsection{Проверка вырожденности треугольника}

Для проверки, не лежат ли вершины треугольника на одной прямой, используется класс \texttt{PointsOnLineChecker}. Алгоритм проверяет, равна ли площадь треугольника нулю.

\begin{algorithm}
\caption{Проверка вырожденности треугольника}
\begin{algorithmic}[1]
\Procedure{IsTriangleDegenerate}{$triangle$}
    \State $a \gets triangle.a$
    \State $b \gets triangle.b$
    \State $c \gets triangle.c$
    \State $area \gets 0.5 \cdot |(b_x - a_x)(c_y - a_y) - (c_x - a_x)(b_y - a_y)|$
    \State \textbf{return} $area = 0$
\EndProcedure
\end{algorithmic}
\end{algorithm}

\subsubsection{Разбиение треугольника медианами}

Для разбиения треугольника на 6 подтреугольников с помощью медиан используется класс \texttt{TriangleMedianCalculator}. Алгоритм вычисляет медианы треугольника и центроид (точку пересечения медиан), а затем формирует 6 подтреугольников.

\begin{algorithm}
\caption{Разбиение треугольника медианами}
\begin{algorithmic}[1]
\Procedure{DivideTriangleByMedians}{$triangle$}
    \State $a \gets triangle.a$
    \State $b \gets triangle.b$
    \State $c \gets triangle.c$
    
    \State $midAB \gets$ Point($(a_x + b_x) / 2$, $(a_y + b_y) / 2$)
    \State $midBC \gets$ Point($(b_x + c_x) / 2$, $(b_y + c_y) / 2$)
    \State $midCA \gets$ Point($(c_x + a_x) / 2$, $(c_y + a_y) / 2$)
    
    \State $centroid \gets$ Point($(a_x + b_x + c_x) / 3$, $(a_y + b_y + c_y) / 3$)
    
    \State $subtriangles \gets$ пустой список
    
    \State Добавить Triangle($a$, $midAB$, $centroid$) в $subtriangles$
    \State Добавить Triangle($a$, $midCA$, $centroid$) в $subtriangles$
    \State Добавить Triangle($b$, $midAB$, $centroid$) в $subtriangles$
    \State Добавить Triangle($b$, $midBC$, $centroid$) в $subtriangles$
    \State Добавить Triangle($c$, $midBC$, $centroid$) в $subtriangles$
    \State Добавить Triangle($c$, $midCA$, $centroid$) в $subtriangles$
    
    \State \textbf{return} $subtriangles$
\EndProcedure
\end{algorithmic}
\end{algorithm}

\subsubsection{Подсчет точек внутри подтреугольников}

Для подсчета количества точек внутри каждого подтреугольника используется класс \texttt{PointInTriangleChecker}. Алгоритм проверяет для каждой точки из множества, лежит ли она внутри каждого подтреугольника.

Алгоритм основан на проверке, с какой стороны от каждого ребра треугольника лежит проверяемая точка. 

1. Функция Sign(point, A, B) (чаще всего) вычисляет «знак» векторного произведения, т. е. фактически то, какой ориентации (левее/правее) находится вектор (A→B) относительно вектора (A→point).  
   • Если это «векторное произведение» положительно, то point лежит по одну сторону от отрезка AB.  
   • Если отрицательно, то по другую сторону.  
   • Если равно нулю, то point лежит прямо на отрезке AB.

2. В алгоритме последовательно вычисляются d1, d2, d3 для трёх рёбер треугольника:
   • d1 = Sign(point, a, b)  
   • d2 = Sign(point, b, c)  
   • d3 = Sign(point, c, a)

3. Затем смотрится, есть ли среди полученных знаков отрицательные (hasNeg) и положительные (hasPos).  
   • hasNeg = true, если среди d1, d2, d3 есть < 0  
   • hasPos = true, если среди d1, d2, d3 есть > 0

4. Если одновременно есть и отрицательные, и положительные значения (hasNeg && hasPos), значит точка «перекрывает» разные ориентации относительно рёбер, то есть она лежит вне треугольника.

5. Если же все знаки одинаковые (все ≥ 0 или все ≤ 0) или какие-то из них равны нулю, но не появляется перемешивания «+» и «−», точка находится внутри треугольника или на его границе.

\begin{algorithm}
\caption{Подсчет точек внутри подтреугольников}
\begin{algorithmic}[1]
\Procedure{CountPointsInsideSubtriangles}{$points$, $subtriangles$}
    \State $counts \gets$ массив из $|subtriangles|$ элементов, инициализированных нулями
    
    \For{each $point$ in $points$}
        \For{$i \gets 0$ to $|subtriangles| - 1$}
            \If{IsPointInTriangle($point$, $subtriangles[i]$)}
                \State $counts[i] \gets counts[i] + 1$
            \EndIf
        \EndFor
    \EndFor
    
    \State \textbf{return} $counts$
\EndProcedure
\end{algorithmic}
\end{algorithm}

\begin{algorithm}
\caption{Проверка, лежит ли точка внутри треугольника}
\begin{algorithmic}[1]
\Procedure{IsPointInTriangle}{$point$, $triangle$}
    \State $a \gets triangle.a$
    \State $b \gets triangle.b$
    \State $c \gets triangle.c$
    
    \State $d1 \gets Sign(point, a, b)$
    \State $d2 \gets Sign(point, b, c)$
    \State $d3 \gets Sign(point, c, a)$
    
    \State $hasNeg \gets (d1 < 0) \lor (d2 < 0) \lor (d3 < 0)$
    \State $hasPos \gets (d1 > 0) \lor (d2 > 0) \lor (d3 > 0)$
    
    \State \textbf{return} $\lnot(hasNeg \land hasPos)$
\EndProcedure

\Procedure{Sign}{$p1$, $p2$, $p3$}
    \State \textbf{return} $(p1_x - p3_x) \cdot (p2_y - p3_y) - (p2_x - p3_x) \cdot (p1_y - p3_y)$
\EndProcedure
\end{algorithmic}
\end{algorithm}

\newpage
\newpage

\section{Примеры}

\subsection{Пример 1}

\textbf{Входные данные:}
\begin{itemize}
    \item Множество точек: $\{(0, 0), (10, 0), (5, 10), (2, 2), (8, 2), (5, 5), (3, 1), (7, 1), (5, 2)\}$
\end{itemize}

\textbf{Результат:}
\begin{itemize}
    \item Треугольник: $\{(0, 0), (10, 0), (5, 10)\}$
    \item Подтреугольник с максимальным количеством точек: содержит 3 точки
    \item Подтреугольник с минимальным количеством точек: содержит 0 точек
    \item Разность: 3
\end{itemize}

\begin{figure}[h]
\centering
\begin{tikzpicture}[scale=0.5]
    % Основной треугольник (зеленый)
    \draw[green, thick] (0,0) -- (10,0) -- (5,10) -- cycle;
    
    % Медианы (синие)
    \draw[blue] (0,0) -- (7.5,5);
    \draw[blue] (10,0) -- (2.5,5);
    \draw[blue] (5,10) -- (5,0);
    
    % Центроид
    \fill[black] (5,3.33) circle (0.2);
    
    % Точки внутри треугольника
    \fill[black] (2,2) circle (0.1);
    \fill[black] (8,2) circle (0.1);
    \fill[black] (5,5) circle (0.1);
    \fill[black] (3,1) circle (0.1);
    \fill[black] (7,1) circle (0.1);
    \fill[black] (5,2) circle (0.1);
    
    % Подтреугольник с максимальным количеством точек (красный)
    \draw[red, thick] (5,0) -- (5,3.33) -- (10,0) -- cycle;
    
    % Подтреугольник с минимальным количеством точек (пурпурный)
    \draw[purple, thick] (5,10) -- (5,3.33) -- (7.5,5) -- cycle;
\end{tikzpicture}
\caption{Пример результата работы алгоритма}
\end{figure}

\end{document}
